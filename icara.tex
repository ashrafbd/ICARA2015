
%\documentclass[conference,draftcls]{IEEEtran}
\documentclass[conference,final]{IEEEtran}

\usepackage{setspace}
%\doublespacing
%\onehalfspacing

\usepackage{cite}
\usepackage[utf8]{inputenc}
\usepackage{cite}
\usepackage{paralist}

\usepackage[pdftex]{graphicx}
\graphicspath{{.}{images/}} 
\DeclareGraphicsExtensions{.pdf,.jpeg,.png,.jpg}

\usepackage{caption}
\usepackage{subcaption}
\usepackage[cmex10]{amsmath}
\usepackage{url}

%\usepackage{algorithmic}
\usepackage{array}
\usepackage{mdwmath}
\usepackage{mdwtab}
%\usepackage{eqparbox}
%\usepackage[tight,footnotesize]{subfigure}
%\usepackage[caption=false]{caption}
%\usepackage[font=footnotesize]{subfig}
\usepackage[caption=false,font=footnotesize]{subfig}
\usepackage{fixltx2e}
\usepackage{stfloats}


% correct bad hyphenation here
\hyphenation{op-tical net-works semi-conduc-tor}


\begin{document}
\title{Intrusion Detection of a Wireless Sensor Network\\Over the Air Update Protocol}
\author{
		\IEEEauthorblockN{	A S M Ashraful Alam\IEEEauthorrefmark{1} and
							David Eyers\IEEEauthorrefmark{1} and 
							Zhiyi Huang\IEEEauthorrefmark{1} and 
						}
	\IEEEauthorblockA{\IEEEauthorrefmark{1}Department of Computer Science, University of Otago, New Zealand, Email: \{aalam,dme,hzy\}@cs.otago.ac.nz} 
}


% use for special paper notices
%\IEEEspecialpapernotice{(Invited Paper)}

\maketitle


\begin{abstract}
Robotics and  Wireless Sensor Network (WSN) collaborations is an emerging research field in which both the technologies can benefit from integrated implementations.
In this paper, an Intrusion Detection System (IDS) for WSN  software update protocol is designed and simulated. 
The IDS is designed to protect software updates using over the air (OTA) update protocols, specifically the deluge protocol.
As an OTA update protocol modifies the running software in a mote, the mote sends necessary information to the sink. 
The IDS in association with the IDS Transport Client (ITC) in the sink analyses the update phenomena of each of the
motes in the network and computes a `Intrusion Warning Score' through an algorithmic function to indicate a possible intrusion i.e., a probable illegitimate update.
\end{abstract}

\begin{keyword}
\kwd{Wireless Sensonrs}
\kwd{WSN}
\kwd{Intrusion Detection System}
\kwd{Security}
\kmw{Robotics}
\end{keyword}

% no keywords
% For peer review papers, you can put extra information on the cover
% page as needed:
% \ifCLASSOPTIONpeerreview
% \begin{center} \bfseries EDICS Category: 3-BBND \end{center}
% \fi
%
% For peerreview papers, this IEEEtran command inserts a page break and
% creates the second title. It will be ignored for other modes.
\IEEEpeerreviewmaketitle %not meant to be here for submission
Security of sensed environmental data external to the robots
%ICARA does accept papers on sensors. With collaborative robotics increasing rapidly, security of inter-robot communication over wireless media is of great importance. Your proposed paper would be of interest to ICARA attendees and we look forward to receiving your paper by 27th October.

\section{Introduction}
%	\subsection{Subsection Heading Here}
%		\subsubsection{Subsubsection Heading Here}
\paragraph{Application of Sensors in Robotics. How important is WSN in robotics}
\paragraph{Why WSN communication in robotics/any application need to be secured. }
If inputs from wireless sensors are forged, then catastrophic errorneous results are expected			
\paragraph{Why it is important to secure updates? How WSN updates can be forged to impede the capabilities of robotics}
\paragraph{Why it is necessary to have IDS not encryption in case of WSN}
\paragraph{Security if WSN and Update needs. what are the problems. How do I plan to handle them (How do I do it). > Contributions} \\
Organisation of the paper

\subsection{Literature Review}
\paragraph{WSN and robotics works and what they mainly address and what they do not}
\paragraph{WSN update security lit review. What approaches are taken in WSN}
\paragraph{IDSs in WSN}
\section{Conclusion}



\section{Introduction}
	\subsection{Subsection Heading Here}
		\subsubsection{Subsubsection Heading Here}
			\paragraph{}
%\begin{figure}[!t]
%\centering
%\includegraphics[width=2.5in]{myfigure}
% where an .eps filename suffix will be assumed under latex, 
% and a .pdf suffix will be assumed for pdflatex; or what has been declared
% via \DeclareGraphicsExtensions.
%\caption{Simulation Results}
%\label{fig_sim}
%\end{figure}

% An example of a double column floating figure using two subfigures.
%\begin{figure*}[!t]
%\centerline{\subfloat[Case I]\includegraphics[width=2.5in]{subfigcase1}%
%\label{fig_first_case}}
%\hfil
%\subfloat[Case II]{\includegraphics[width=2.5in]{subfigcase2}%
%\label{fig_second_case}}}
%\caption{Simulation results}
%\label{fig_sim}
%\end{figure*}
% Note that often IEEE papers with subfigures do not employ subfigure
% captions (using the optional argument to \subfloat), but instead will
% reference/describe all of them (a), (b), etc., within the main caption.

% An example of a floating table. Note that, for IEEE style tables, the 
% \caption command should come BEFORE the table. Table text will default to
% \footnotesize as IEEE normally uses this smaller font for tables.
% The \label must come after \caption as always.
%
%\begin{table}[!t]
%% increase table row spacing, adjust to taste
%\renewcommand{\arraystretch}{1.3}
% if using array.sty, it might be a good idea to tweak the value of
% \extrarowheight as needed to properly center the text within the cells
%\caption{An Example of a Table}
%\label{table_example}
%\centering
%% Some packages, such as MDW tools, offer better commands for making tables
%% than the plain LaTeX2e tabular which is used here.
%\begin{tabular}{|c||c|}
%\hline
%One & Two\\
%\hline
%Three & Four\\
%\hline
%\end{tabular}
%\end{table}
% Note that IEEE does not put floats in the very first column - or typically
% anywhere on the first page for that matter. Also, in-text middle ("here")
% positioning is not used. Most IEEE journals/conferences use top floats
% exclusively. Note that, LaTeX2e, unlike IEEE journals/conferences, places
% footnotes above bottom floats. This can be corrected via the \fnbelowfloat
% command of the stfloats package.

%\section*{Acknowledgment}
% trigger a \newpage just before the given reference
% number - used to balance the columns on the last page
% adjust value as needed - may need to be readjusted if
% the document is modified later
%\IEEEtriggeratref{8}
% The "triggered" command can be changed if desired:
%\IEEEtriggercmd{\enlargethispage{-5in}}

% references section
\bibliographystyle{IEEEtran}
\bibliography{icara}



% that's all folks
\end{document}


